\documentclass{article}
\usepackage{xcolor}
\usepackage{tabularx}
\usepackage{booktabs}
\usepackage[normalem]{ulem}

\title{SE 3XA3: Problem Statement\\Sketchy Super Mario Bros.}

\author{Team \#1, Wario's Miners
		\\ Rylan Sykes (sykesr)
		\\ Jason Nam and (namy2)
		\\ Kristine Uchendu (uchenduc)
}

\date{}

%\input{../Comments}

\begin{document}

\begin{table}[hp]
\caption{Revision History} \label{TblRevisionHistory}
\begin{tabularx}{\textwidth}{llX}
\toprule
\textbf{Date} & \textbf{Developer(s)} & \textbf{Change}\\
\midrule
28/01/22 & Kristine Uchendu, Rylan Sykes, and Jason Nam & Final revisions made\\
\bottomrule
\end{tabularx}
\end{table}

\newpage

\maketitle

\section{Problem Statement}
    There are not many games that give us as much nostalgia as the original Super Mario Bros. 16-bit game. It is a shame that many lovers of the game have no means to play the game today. It is our job to bring back and recreate the beloved game to the wider audience. We will implement the game to the desktop environment with {\color{red}\sout{enhanced features and improved interfaces to provide a means for entertainment.} previously unimplemented features as well as improved graphics, sound, and general user experience.}

\section{Why is the problem important?}
    As society enters the 3rd year of a pandemic filled with copious amounts of working from home and spending time on our personal computers, it’s essential for those who choose to relax through video games to have the opportunity to do so. Our problem pertains to the huge audience of PC gamers who play PC games for leisure,  {\color{red}\sout{de-stress} stress-relief}, monetary gain, and many more reasons. On top of the general gaming crowd, our product allows those who played retro games on their original consoles to relive the experience that is not suited for modern day hardware. This allows the original  {\color{red}\sout{gamers} players of the game} to have a nostalgic experience while disregarding the flaws and drawbacks of older hardware, and embracing new features powered by current day technologies.

\section{Context}
    The stakeholders of this project are the end-users (the players of the game) and developers (us or any developer that may contribute to this game in future). It is very likely that the end-users of the game are going to be people who have already played the game in some form and would like a desktop version to play in their spare time. Because the original project is written in Java, in order to play the game, end-users must be able to run Java code which can be done on any machine with a Java Virtual Machine. {\color{red} Something that we will need to keep in mind when considering our end-user as a stakeholder, is how we will make the game experience enjoyable, as well as how we will determine that we succeeded in doing that. In terms of future developers as a stakeholder, something we will need to keep in my during this project is programming for change. If code is written that is hard to understand and maintain then it makes it difficult for future developers to work on it. If know one is able to look at the code, understand it, and contribute, then updates never happen and the quality of the product ends up suffering, which also impacts our end-users.}
    
\end{document}